\prefacesection{Acknowledgments}

This work was deeply influenced by conversations and friendships with many kind and smart people, to whom I owe so much.

Gretchen Daily welcomed me to her lab in the Biology Department in early 2016. She has been a champion and an advocate for me every day since, and I am profoundly grateful for the opportunities she has created for me. I am inspired by her commitment to understanding the value of nature—from all perspectives—and to finding opportunities to conserve and cherish our planet's wondrous biodiversity. We've always shared the commitment; she taught me how to translate it into action. I've learned so much from Gretchen: how to communicate my work with passion and precision; how to build an organization dedicated to inclusive, science-driven conservation; the value of leading through empowering others; and how to meet every person where they are. As my primary adviser, Gretchen was solely focused on my development as a scientist, ensuring I was working on problems I was passionate about and that I had access to the people and resources that would help me grow. I benefit tremendously as a result. Thank you.

I have had the honor and the privilege of being advised by Erin Gilmour Mordecai and Chris Field, two thoughtful, critical and generous professors who inspire and challenge me in every discussion we have. When most of my early research focused on finding empirical patterns in ecological data, Erin encouraged me to dig deeper and deeper into the biological mechanisms that might underpin them, helping me think more clearly and test assumptions more rigorously than I would have otherwise. Chris has a wealth of knowledge and experience as a scientist and as an adviser. I am regularly amazed by his ability to ask the right question or provide the right recommendation to help me think through an issue more clearly and guiding me towards novel research frontiers. They have both shaped my perspective as a scientist, and I am grateful for their advice.

In the Biology Department, I wish to express my gratitude to Hal Mooney and Rodolfo Dirzo. Both Hal and Rodolfo have made themselves available as mentors throughout my time at Stanford, and have provided insights and encouragement that I value deeply. They are both role models for me as global ecologists, and it's been a thrill to spend time with them.

I am fortunate to be affiliated with the Stanford Center for Conservation Biology. I built strong professional and personal relationships there with Jeff Smith, Nick Hendershot, Kelly Langhans, Greg Bratman, Hannah Frank, and Jon Flanders. I am particularly grateful for the time spent with Jeff, Nick, and Kelly, as we had the opportunity to grow together as Ph.D. students and to constantly learn from each other. I had a blast.

I spent many hours learning from the scientists at the Natural Capital Project, and probably even more hours socializing with them. Thank you to the NatCap team for welcoming me and teaching me new ways to think about the value of nature. Special thanks are due to Becky Chaplin-Kramer, Lisa Mandle, Rich Sharp, Ben Bryant, Marcelo Guevara, Morgan Kain, Adrian Vogl, Sarah Cafasso, Perrine Hamel, Charlie Weil, James Douglass, Steve Polasky and Mary Ruckelshaus.

Thank you to all of the collaborators who helped design, plan and execute the mosquito vector field surveys in Costa Rica and Peru, which are reported in \textit{Chapter 4}. This includes Meghan Howard, Wily Lescano, Luis Fernandez, Stephanie Montero Trujillo, Ricardo Gamboa, Mileyka Santos Gaitán, Denis Navarro, Eloy Inisuy, Paul Sairitupa, Joel Sajami, and Jamieson O'Marr.

I am grateful for the support of Stanford University, which provided access to a unique array of educational resources. I spent many hours in the collections of the Branner Earth Sciences Library, the Bowes Art \& Architecture Library and the David Rumsey Map Center. Libraries are tremendously valuable and often underutilized resources, and I appreciate the University's commitment to archiving, indexing and making accessible their unique and expansive collections. I have also benefit from access to resources like the Hume Center for Writing and Speaking and the Stanford Geospatial Center, which helped me grow as a writer, an analyst and as a scholar.

Prior to enrolling at Stanford, I worked as a staff research assistant in the Department of Global Ecology at the Carnegie Institution for Science. I owe much to Greg Asner and Robin Martin, who gave me opportunity after opportunity to learn and grow as an ecologist and as a technologist. I was able to travel the world with the Carnegie Airborne Observatory, spending hundreds of hours surveying forests from the sky. I've flown transects from the Andes to the Amazon, witnessing both the unparalleled beauty of the transition from treeline to the lowlands, and the rapid loss of forests to fires, gold mining, and logging. The experience richly colored my understanding of the earth system: as beautiful, complex, very big, and changing in ways that are hard to convey with numbers alone.

I was also fortunate to meet and learn from an outstanding group of scientists during my time at the Department of Global Ecology. This includes Claire Baldeck, Jomar Barbosa, Joe Berry, Joe Boardman, Paulo Brando, Phil Brodrick, Loreli Carranza-Jiménez, Dana Chadwick, Cecilia Chavana-Bryant, John Clark, Andrew Davies, Shane Easter, Michael Eastwood, Jean-Baptiste Féret, Emily Francis, Mark Helmlinger, Mona Houcheime, James Jacobson, Jen Johnson, Ty Kennedy-Bowdoin, Dave Knapp, Dave Marvin, Joe Mascaro, Kelly McManus, Elsa Ordway, Ted Raab, Elif Tassar, Phil Taylor, Todd Tobeck, Raul Tupayachi, Nick Vaughn, and Parker Weiss. I owe yet more thanks to Chris Field, then the director of the department, who fostered an open and welcoming environment where I was empowered to learn and grow.

Thank you to my parents, Rhonda and Robert Anderson, for loving me and supporting my passion for ecology. You've taught me so much, and I cherish you both.

I am grateful for my wife Helen, who shared her kindness and her love with me before and throughout my time at Stanford. She taught me a very important lesson soon after we met: that interested people are interesting. I reflect on this often, which reminds me that listening is both an expression of kindness and caring for others as well as a means to grow as a person. She reinforces my appreciation of libraries through her work, and demonstrates an inspiring example of the value of service to others. Thank you for sharing so much with me.

This work was supported by the Bing-Mooney Fellowship in Environmental Science through the Department of Biology at Stanford University. Funding for field work came through two grants: ``New Science and Technology for Measuring Biodiversity" from the Moore Family Foundation, and ``PRO Agua: Resiliencia Natural en la Amazonia" from the Gordon and Betty Moore Foundation.