\chapter{Glossary}

\begin{itemize}

\item \textit{Biodiversity pattern}: recurring and structured variation in the distributions of genes, species, communities and ecosystems.

\item \textit{Continuous measurements}: Earth observation measurements mapping the full geographical extent of a region without gaps.

\item \textit{Data dimensionality}: the minimum number of free variables needed to represent data without information loss \cite{Camastra2003-dj}.

\item \textit{Discrete measurements}: Earth observation measurements mapping specific areas that do not cover the full geographical extent of a region.

\item \textit{Ecological processes}: Activities that result from interactions among organisms and between organisms and their environment\cite{Martinez1996-vg}.

\item \textit{Earth observation sensor}: spaceborne or airborne instruments (e.g., a camera or radar) that record the electromagnetic radiation emitted or reflected by the landscape \cite{Campbell2011-sa}.

\item \textit{Extent}: the range over which a pattern or process occurs or is expected to occur \cite{Nekola1999-ks}, such as a species fundamental niche, or the total area measured by an EO sensor.

\item \textit{Grain size}: the size of the smallest individual unit of measurement \cite{Jensen1987-nd}, such as a plot or transect in ecology, or the ground sampling distance of an Earth Observation sensor.

\item \textit{Multi-sensor fusion}: integrating measurements from multiple sensors with complementary spatial and temporal characteristics to characterise a single pattern \cite{Hilker2009-aw}.

\item \textit{Radiometric calibration}: the conversion of raw image data (e.g. in digital number format) to units of absolute radiance (e.g. in $W\, m^{-2}\, sr^{-1}\, {\mu}m^{-1}$) to standardise data from multiple sensors into a common scale \cite{Chander2009-cn}.

\item \textit{Sensor fidelity}: the ability of a sensor to discriminate between land surface properties, and to discriminate signal from noise across the dynamic range of the sensor \cite{Campbell2011-sa}.

\item \textit{Sensor type}: general classifications of EO sensors based on the range of electromagnetic radiation measured, and how it was measured. Sensors are typically classified as active (i.e., sensors that emit their own energy, then record the reflection of that energy by the surface) or passive (i.e., sensors that measure energy emitted by the surface, not generated by the sensor). Radar sensors (e.g. Sentinel-1) are an example of active microwave (1 $mm$ to 1 $m$) sensors. Multispectral sensors (e.g., Landsat) are an example of passive optical sensors that measure a range of typically visible (0.38–0.78 ${\mu}m$) to near-infrared (0.78–1.3 ${\mu}m$) or shortwave-infrared (1.3–3 ${\mu}m$) wavelengths \cite{Campbell2011-sa}.

\end{itemize}